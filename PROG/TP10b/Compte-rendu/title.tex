\documentclass{article}
%packages
\usepackage{graphicx}
\usepackage[latin1]{inputenc}
\usepackage[T1]{fontenc}
\usepackage[frenchb]{babel}
\usepackage[a4paper]{geometry}

\begin{document}
%title
\begin{titlepage}
	\vspace{-20px}
	\begin{tabular}{l}
		\textsc{Blin} S\'ebastien\\
		\textsc{Collin} Pierre-Henri
	\end{tabular}
	\hfill \vspace{10px}\includegraphics[scale=0.1]{esir}\\
	\vfill
	\begin{center}
		\Huge{\'Ecole sup\'erieure d'ing\'enieurs de Rennes}\\
		\vspace{1cm}
		\LARGE{1\`ere Ann\'ee}\\
		\large{Parcours Informatique}\\
		\vspace{0.5cm}\hrule\vspace{0.5cm}
		\LARGE{\textbf{Compte-rendu}}\\
		\Large{TP n°2 : Un vecteur g\'en\'erique}
		\vspace{0.5cm}\hrule
		\vfill
		\vfill
	\end{center}
	\begin{flushleft}
		\Large{Sous l'encadrement de~:}\\
		\vspace{0.2cm}
		\large{{Lamarche} Fabrice}
	\end{flushleft}
	\vfill
\end{titlepage}

\section{Une classe de vecteur g\'en\'erique}
\subsection{Param\`etres g\'en\'eriques}
Le seul param\`etre g\'en\'erique de la classe Vecteur est T *m\_vecteur; 
\subsection{Contraintes associ\`ees}
Ce type T doit red\'efinir plusieurs op\'erateurs notamment l'op\'erateur + qui est utilis\'e (ou encore l'op\'erateur *).
\subsection{Addition de 2 std::string}
L'addition de 2 std::string signifie une concat\'enation.
\section{Des fonctions g\'en\'eriques}
\subsection{Nouvelles contraintes}
Le type T doit maintenant red\'efinir les op\'erateurs << et >>
\subsection{Compilateur et g\'en\'ericit\'e}
Une multiplication scalaire de std::string ne veut rien dire. Il y a une erreur car l'operateur * n'est pas d\'efini pour le type std::string.\\
Le compilateur compile chaque fonction et chaque classe g\'en\'erique pour chaque type utilis\'e (et effectue les v\'erifications \`a chaque fois).
\section{Les types}
\subsection{Multiplication d'un int et d'un float}
Pour le moment on ne peut faire des multiplication que d'un type (au-dessus de l'operator* il y a template<class T>). Pour r\'egler ce probl\`eme i suffit de mettre 2 classes dans le template.
\subsection{a*b ; b*a}
Dans notre multiplication, notre r\'esultat est cast\'e dans le type de la premi\`ere op\'erande (c'est-\`a-dire a*b sera du type de a, b*a sera du type b)
\end{document}
